\documentclass[man]{apa6}
\usepackage{lmodern}
\usepackage{amssymb,amsmath}
\usepackage{ifxetex,ifluatex}
\usepackage{fixltx2e} % provides \textsubscript
\ifnum 0\ifxetex 1\fi\ifluatex 1\fi=0 % if pdftex
  \usepackage[T1]{fontenc}
  \usepackage[utf8]{inputenc}
\else % if luatex or xelatex
  \ifxetex
    \usepackage{mathspec}
  \else
    \usepackage{fontspec}
  \fi
  \defaultfontfeatures{Ligatures=TeX,Scale=MatchLowercase}
\fi
% use upquote if available, for straight quotes in verbatim environments
\IfFileExists{upquote.sty}{\usepackage{upquote}}{}
% use microtype if available
\IfFileExists{microtype.sty}{%
\usepackage{microtype}
\UseMicrotypeSet[protrusion]{basicmath} % disable protrusion for tt fonts
}{}
\usepackage{hyperref}
\hypersetup{unicode=true,
            pdftitle={Supplemental Materials for How Many Psychologists Use Questionable Research Practices? Estimating the Population Size of Current QRP Users},
            pdfauthor={Nicholas W. Fox, Nathan Honeycutt, \& Lee Jussim},
            pdfkeywords={QRPs, Questionable Research Practices, Replication Crisis, Social
Networks},
            pdfborder={0 0 0},
            breaklinks=true}
\urlstyle{same}  % don't use monospace font for urls
\usepackage{graphicx,grffile}
\makeatletter
\def\maxwidth{\ifdim\Gin@nat@width>\linewidth\linewidth\else\Gin@nat@width\fi}
\def\maxheight{\ifdim\Gin@nat@height>\textheight\textheight\else\Gin@nat@height\fi}
\makeatother
% Scale images if necessary, so that they will not overflow the page
% margins by default, and it is still possible to overwrite the defaults
% using explicit options in \includegraphics[width, height, ...]{}
\setkeys{Gin}{width=\maxwidth,height=\maxheight,keepaspectratio}
\IfFileExists{parskip.sty}{%
\usepackage{parskip}
}{% else
\setlength{\parindent}{0pt}
\setlength{\parskip}{6pt plus 2pt minus 1pt}
}
\setlength{\emergencystretch}{3em}  % prevent overfull lines
\providecommand{\tightlist}{%
  \setlength{\itemsep}{0pt}\setlength{\parskip}{0pt}}
\setcounter{secnumdepth}{0}
% Redefines (sub)paragraphs to behave more like sections
\ifx\paragraph\undefined\else
\let\oldparagraph\paragraph
\renewcommand{\paragraph}[1]{\oldparagraph{#1}\mbox{}}
\fi
\ifx\subparagraph\undefined\else
\let\oldsubparagraph\subparagraph
\renewcommand{\subparagraph}[1]{\oldsubparagraph{#1}\mbox{}}
\fi

%%% Use protect on footnotes to avoid problems with footnotes in titles
\let\rmarkdownfootnote\footnote%
\def\footnote{\protect\rmarkdownfootnote}


  \title{Supplemental Materials for `How Many Psychologists Use Questionable
Research Practices? Estimating the Population Size of Current QRP Users'}
    \author{Nicholas W. Fox\textsuperscript{1}, Nathan Honeycutt\textsuperscript{1},
\& Lee Jussim\textsuperscript{1}}
    \date{}
  
\shorttitle{SM: How Many Psychologists Use QRPs}
\affiliation{
\vspace{0.5cm}
\textsuperscript{1} Rutgers University}
\keywords{QRPs, Questionable Research Practices, Replication Crisis, Social Networks\newline\indent Word count: 4,492}
\usepackage{csquotes}
\usepackage{upgreek}
\captionsetup{font=singlespacing,justification=justified}

\usepackage{longtable}
\usepackage{lscape}
\usepackage{multirow}
\usepackage{tabularx}
\usepackage[flushleft]{threeparttable}
\usepackage{threeparttablex}

\newenvironment{lltable}{\begin{landscape}\begin{center}\begin{ThreePartTable}}{\end{ThreePartTable}\end{center}\end{landscape}}

\makeatletter
\newcommand\LastLTentrywidth{1em}
\newlength\longtablewidth
\setlength{\longtablewidth}{1in}
\newcommand{\getlongtablewidth}{\begingroup \ifcsname LT@\roman{LT@tables}\endcsname \global\longtablewidth=0pt \renewcommand{\LT@entry}[2]{\global\advance\longtablewidth by ##2\relax\gdef\LastLTentrywidth{##2}}\@nameuse{LT@\roman{LT@tables}} \fi \endgroup}


\DeclareDelayedFloatFlavor{ThreePartTable}{table}
\DeclareDelayedFloatFlavor{lltable}{table}
\DeclareDelayedFloatFlavor*{longtable}{table}
\makeatletter
\renewcommand{\efloat@iwrite}[1]{\immediate\expandafter\protected@write\csname efloat@post#1\endcsname{}}
\makeatother
\usepackage{tcolorbox}

\authornote{Department of Psychology, Rutgers University,
Piscataway NJ 08854.

Correspondence concerning this article should be addressed to Nicholas
W. Fox, 53 Avenue E, Room 429, Piscataway NJ 08854. E-mail:
\href{mailto:nwf7@psych.rutgers.edu}{\nolinkurl{nwf7@psych.rutgers.edu}}}

\abstract{
These are the supplemental materials for ``How Many Psychologists Use
Questionable Research Practices? Estimating the Population Size of
Current QRP Users''. This file may change as more information is added.


}

\usepackage{amsthm}
\newtheorem{theorem}{Theorem}[section]
\newtheorem{lemma}{Lemma}[section]
\theoremstyle{definition}
\newtheorem{definition}{Definition}[section]
\newtheorem{corollary}{Corollary}[section]
\newtheorem{proposition}{Proposition}[section]
\theoremstyle{definition}
\newtheorem{example}{Example}[section]
\theoremstyle{definition}
\newtheorem{exercise}{Exercise}[section]
\theoremstyle{remark}
\newtheorem*{remark}{Remark}
\newtheorem*{solution}{Solution}
\begin{document}
\maketitle

\section{Supplemental Materials}\label{supplemental-materials}

\subsection{Study Preregistration}\label{study-preregistration}

This study was preregistered on May 15th, 2017 and can be found at
\url{https://osf.io/xu25n/}. The final study deviated from the
preregistration in three places.

First, the title changed from the working title of \enquote{How Many
Psychologists Use QRPs? A Social Network Approach to Estimating Hidden
Population Size} to \enquote{How Many Psychologists Use Questionable
Research Practices? Estimating the Population Size of Current QRP
Users}. This change was made for two reasons. First, we did not want to
assume all readers had an innate knowledge of what \enquote{QRPs} meant,
so we wanted to avoid using the abbreviation solely in the title.
Second, as the paper has multiple estimates, and not just social network
estimates, we wanted to broaden the scope of the title to be inclusive
of the direct and indirect estimates.

Second, Nathan Honeycutt was added as an author. His contribution to the
final paper was in providing constructive comments during the drafting
of the manuscript and proofreading the final document. Due to his
substantial contributions to the readability and flow of the final
document, authhorship was warrented.

Third, we preregistered that this work would be conducted using two
surveys, when instead we used three surveys for this project.
Originally, the direct estimate and game of contacts methods would have
followed the innoculus list condition of the UCT. However, upon further
research, the primary author wanted to avoid potential priming effects
of one estimate on how participant's responded to the subsequent direct
estimate. Therefore, the decision was made to split the direct estimate
(and the subsequent game of contacts method) into a new survey (Survey
3). Therefore, the final survey methodology consisted of three surveys,
with one containing the innocuous UCT condition, one containing the
sensitive UCT condition, and one constaining the direct estimate.

\subsection{Questionable Research Practice
Definition}\label{questionable-research-practice-definition}

Questionable Research Practices were defined to participants in the
following way: ``Some of the following questions may ask you about
Questionable Research Practices, or QRPs. We define a Questionable
Research Practice (QRP) as one of the 9 following behaviors when
collecting, analyzing, or publishing scientific data performed in the
past 12 months:

\begin{itemize}
\item Deciding whether to collect more data after looking to see whether the results were statistically significant.
\item Deciding whether to exclude data after looking at the impact of doing so on the results. 
\item Stopping collecting data earlier than planned because one found the result one had been looking for. 
\end{itemize}

or, in a paper published in the past 12 months:

\begin{itemize}
\item Did not report all of the study's dependent measures. 
\item Did not report all of a study's conditions.
\item "Rounding off" a p value (e.g., reporting that a p value of 0.054 is less than 0.05).
\item Selectively reporting studies that "worked". 
\item Reporting an unexpected finding as having been predicted from the start. 
\item Claiming that results are unaffected by demographic variables (e.g., gender) when one is actually unsure (or knows that they do).
 \end{itemize}

This definition will be available to you by hovering over any question
with an underlined \enquote{\underline{QRP}}

Try hovering over the highlighted text now. When you understand the
definitions listed above, please continue by clicking the "
\textgreater{}\textgreater{}\textgreater{}" button below."

The underlined \enquote{QRP} in the above definition, when moused-over
by the participant, produced a pop-out window in Qualtrics that provided
the following information: ``A QRP is one of the following:

\begin{itemize}
\item Deciding whether to collect more data after looking to see whether the results were statistically significant.
\item Deciding whether to exclude data after looking at the impact of doing so on the results.
\item Stopping collecting data earlier than planned because one found the result one had been looking for.

or, in a paper published in the past 12 months:
\item Did not report all of a study's dependent measures.
\item Did not report all of a study's conditions.
\item "Rounding off" a p value (e.g., reporting that a p value of 0.054 is less than 0.05).
\item Selectively reporting studies that "worked".
\item Reporting an unexpected finding as having been predicted from the start.
\item Claiming that results are unaffected by demographic variables (e.g., gender) when one is actually unsure (or knows that they do)."
\end{itemize}

\newpage

\begingroup
\setlength{\parindent}{-0.5in} \setlength{\leftskip}{0.5in}

\hypertarget{refs}{}

\endgroup


\end{document}
